\documentclass[12pt, a4paper, twoside]{book}
\usepackage[utf8]{inputenc} % Aceptar diferentes tipos de codificación de caracteres de entrada (en este caso usamos la codificación Unicode UTF-8)
%\usepackage{natbib}
\usepackage{listings}
\usepackage{eurosym}
\usepackage[spanish]{babel}
\usepackage{titlesec}
\usepackage{graphicx} % Soporte aumentado para gráficos 
\usepackage{float}
\usepackage{hyperref} % Para manejar referencias cruzadas. P.ej. añadir hiperenlaces al índice
\usepackage{caption}
\usepackage{setspace}
\usepackage{color}
\usepackage[a4paper, top=3.5cm, bottom=3.5cm, left=3cm, right=3cm]{geometry}
\spacing{1.5}
\setcounter{secnumdepth}{4}
\setlength{\parindent}{12pt}
\titleformat{\paragraph}
{\normalfont\normalsize\bfseries}{\theparagraph}{1em}{}
\titlespacing*{\paragraph}
{0pt}{3.25ex plus 1ex minus .2ex}{1.5ex plus .2ex}

%\usepackage{inconsolata}
%
%\usepackage[T1]{fontenc}
%
%\definecolor{pblue}{rgb}{0.13,0.13,1}
%\definecolor{pgreen}{rgb}{0,0.5,0}
%\definecolor{pred}{rgb}{0.9,0,0}
%\definecolor{pgrey}{rgb}{0.46,0.45,0.48}
%
%\lstset{language=Java,
%	showspaces=false,
%	showtabs=false,
%	breaklines=true,
%	showstringspaces=false,
%	breakatwhitespace=true,
%	commentstyle=\color{pgreen},
%	keywordstyle=\color{pblue},
%	stringstyle=\color{pred},
%	basicstyle=\ttfamily,
%	moredelim=[il][\textcolor{pgrey}]{$$},
%	moredelim=[is][\textolor{pgrey}]{\%\%}{\%%}
%}


\begin{document}	
	
	\thispagestyle{empty} 	
	%%%%%%%%%%%%%%%%%%%%%%%%%%%%%%%%%%%%%%%%%%%%%%%%%%%%%%%%%%%%%%%%%%%%%%%%%%%%%%%%
	% PORTADA
	%%%%%%%%%%%%%%%%%%%%%%%%%%%%%%%%%%%%%%%%%%%%%%%%%%%%%%%%%%%%%%%%%%%%%%%%%%%%%%%%
	
	\begin{center}		
		\includegraphics[width=15cm]{Imagenes/Simbolo_logo_UDC.png}
	\end{center}
	
	% Lista de tamaños: \Huge, \huge, \LARGE, \Large, \large, \small, \footnotesize, \tiny
	\vspace{2cm}
	
	\begin{center}		
		{\textbf{ FACULTADE DE INFORMÁTICA}}
			
		\vspace{1cm}
		\LARGE{ TRABALLO FIN DE MÁSTER }	\\
		\LARGE{ MÁSTER UNIVERSITARIO EN INGENIERÍA INFORMÁTICA } \\
		\vspace{1cm}	
		\LARGE{\textbf{ Aplicación web para a xestión de menús domésticos con servizos nutricionais : Eat Fit Week! }}
	\end{center}
	
	\vspace{2cm}
	\hfill \textbf{Autor: \textit{Elías Ferreiro Borreiros}}
	
	
	\hfill \textbf{Director: \textit{Juan José Sánchez Penas}} 
	
	
	\hfill A Coruña, Agosto, 2019					
	
	
	\clearpage
	
	\vspace*{\fill}
	\hfill A mi familia
	\vspace*{\fill}
	
	
	\clearpage
	
	\begin{center}
		\LARGE{\textbf{ AGRADECIMIENTOS }}	
	\end{center}	
	A mi familia y a mis amigos, por su apoyo incondicional y su paciencia.\\
	A Esther por estar ahí para mí incluso en los días más duros o sobre todo en ellos.
	
	\clearpage
	
	\begin{center}
		\LARGE{\textbf{ RESUMEN }}	
	\end{center}
	Hoy en día, con el cambio en los estilos de vida de las personas y tendiendo hacia unas costumbres más sedentarias, hay una mayor necesidad de enfocarse en una dieta equilibrada y saludable.
	Para ello, se han desarrollado muchos sistemas webs y móviles para la gestión de comidas y de sus valores nutricionales.	Sin embargo, analizando esos sistemas, vemos que tienen un error en su planteamiento al inundar a los usuarios con formularios sobrecargados y repletos de información innecesaria. 
	El otro problema principal de estos sistemas es la cantidad exagerada de trabajo manual que debe hacer el usuario antes de poder disfrutar de la funcionalidad principal. 

	Para resolver todo esto, hemos decidido plantear el desarrollo de una aplicación que solvente estos problemas y ofrezca una funcionalidad que no disponen los competidores : el análisis nutricional dinámico de las comidas planificadas para la semana configurable por el usuario.
	El usuario dispondrá de unos ciertos parámetros para la planificación de sus menús: cantidad de calorías, proteínas, grasas ...
	Una vez configurados, a medida que se vayan añadiendo platos al menú semanal se verificarán estos parámetros para indicar al usuario si está cumpliendo con sus especificaciones o si se está sobrepasando.

	A mayores permitiremos la gestión de las entidades necesarias para esta planificación: ingredientes, platos, menús ... 
	Esto se hará siguiendo la filosofía inicial del proyecto: simplificar la entrada lo más posible y disminuir el esfuerzo requerido por el usuario. 
	Para esto llamaremos a servicios externos que nos permitirán estimar las características nutricionales de los ingredientes de forma que el usuario no tendrá que indicar esos datos y permitiremos con cada registro de usuario el alta automática de unos ingredientes base utilizables en la mayoría de recetas que agilizarán la configuración necesaria de un nuevo perfil para permitir disfrutar al máximo al usuario de las funcionalidades realmente importantes desde el momento más temprano posible.
	
	\clearpage
	
	\textbf{Título:} Aplicación web para a xestión de menús domésticos con servizos nutricionais
	\\
	\textbf{Autor:} Elías Ferreiro Borreiros
	\\
	\textbf{Tutor/Director:} Juan José Sánchez Penas
	
	
	\textbf{Palabras clave:} Java EE, POJO, Maven, Angular JS, Spring, Hibernate, Web, MySQL, Tarea, Lista, Contexto. 
	
	
	\renewcommand{\contentsname}{Índice de contenidos}
	\renewcommand{\listfigurename}{Índice de figuras}
	\renewcommand{\listtablename}{Índice de tablas}
	
	\tableofcontents % indice de contenidos
	
	\listoffigures % indice de figuras
	
	\listoftables % indice de tablas
	
	\clearpage
	
	\chapter{INTRODUCCIÓN}	
	\section{Determinación de la situación actual}
	\section{Alcance y objetivos}
	\chapter{BASE TECNÓLOGICA}
	\section{Lenguajes}
	\subsection{Java SE 8}
	\subsection{HTML}
	\subsection{CSS}
	\section{Frameworks y librerías}
	\subsection{Core}
	\subsubsection{Spring}
	\subsubsection{Hibernate}	
	\section{Web}
	\subsubsection{Angular JS}
	\subsubsection{Bootstrap}     
	\subsection{Pruebas}
	\subsubsection{JUnit}
	\subsubsection{Spring Test Context}
	\subsubsection{Eclemma}	
	\subsection{Protocolos}
	\section{Hypertext Transfer Protocol o HTTP}
	\subsection{Herramientas de Desarrollo}
	\section{Maven}
	
	\subsection{Servidores de Aplicaciones}
	
	\subsection{Sistemas de Gestión de Bases de Datos}
	
	\subsection{Herramientas de apoyo}
  
	\chapter{INTRODUCCIÓN AL DESARROLLO REALIZADO}
	\section{Introducción}

	\section{Tecnologías}
		
	\section{Metodología e Iteraciones}

	\subsection{Proceso Unificado}

	\subsubsection{Fases del proceso unificado}
	
	\subsubsection{Fase de Inicio}
	
	\subsection{Iteraciones}

	\chapter{PLANIFICACIÓN Y ANÁLISIS DE COSTES}
	
	\section{Análisis de viabilidad}
	
	\section{Planificación}

	\subsection{Planificación previa}
	
	\subsection{Iteraciones}
	
	\subsection{Diagrama de Gantt}

	\chapter{REQUISITOS DEL SISTEMA}
	\section{Introducción}
	\section{Actores}
	\section{Casos de Uso}
	\subsection{Casos de uso comunes}		 
	\section{Modelo de Casos de uso}
	\subsection{Casos de uso comunes}
	\subsection{Casos de uso usuario}
	\subsection{Casos de uso administrador}
	\chapter{DISEÑO DE LA APLICACIÓN}
	\section{Introducción y Objetivos}
	\section{Resumen de Patrones usados\cite{Patrones}}
	
	\section{Arquitectura general}
	\section{SUBSISTEMA POJO-MODELUTIL}
	\subsection{Objetivos}  	
	\subsection{Utilidades}
	\subsubsection{Excepciones}
	
	\subsubsection{DAO Genérico}
	
	\section{SUBSISTEMA APLICACIÓN}

	\subsection{Arquitectura}
	
	\subsection{Modelo del dominio}    
	
	\subsubsection{Diagrama de Entidades}
	
	\subsubsection{Modelo de Datos}

	\subsubsection{Diagrama de Entidad Relación}	
	
	\subsubsection{Consideraciones Adicionales de Modelado de Datos}
	
	\subsection{Capa de Acceso a Datos}
	
	\subsection{Capa Servicios del Modelo}
	
	\subsection{Capa Web}
	
	\subsubsection{Sistema de autenticación}

	\begin{lstlisting}
	\end{lstlisting}
	
	\subsubsection{Elementos empleados en las páginas}
	
	\subsubsection{Estructura de pantallas}
	
	\chapter{IMPLEMENTACIÓN}
	\section{Software requerido}

	\section{Estructura}
	
	\subsection{Instrucciones de compilación}
	
	% Arracamos Tomcat yendo a la carpeta %apache-tomcat-<numero_version> y ejecutamos el achivo startup que se encuentra en la carpeta bin.
	%Ahora podemos utilizar la aplicación en el navegador en localhost:8080/menus.
	\chapter{PRUEBAS}
	\section{Introducción}
	
	\section{Pruebas Unitarias}

	\section{Pruebas de Integración}
	
	\chapter{CONCLUSIONES Y FUTURAS LÍNEAS DE TRABAJO}
	\section{Conclusiones}
	
	\section{Futuras Líneas de Trabajo}
	
	\renewcommand{\bibname}{Enlaces de interés}
	\begin{thebibliography}{99}

	\end{thebibliography}
	\chapter{ACRÓNIMOS}

	\appendix
	\chapter{APÉNDICE}
	\section{Instalación del Software}

	
	% Arracamos Tomcat yendo a la carpeta %apache-tomcat-<numero_version> y ejecutamos el achivo startup que se encuentra en la carpeta bin.
	%Ahora podemos utilizar la aplicación en el navegador en localhost:8080/menus.
	\section{Contenido del CD}
	En el CD incluiremos lo siguiente: El código de la aplicación desarrollada, esta memoria y el resumen del proyecto.
	\section{Manual de Usuario}
	En esta sección se explicará el manejo de la aplicación web desarrollada.
		
	
\end{document}
